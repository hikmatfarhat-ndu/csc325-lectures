
\documentclass{article}
\usepackage{gastex}
\usepackage[usenames,dvips]{color}
\usepackage{amsmath}
\usepackage{amssymb}
\usepackage{times}
%\newcommand{\SpaceChar}{\ensuremath{\text{\textvisiblespace}}}
\newcommand{\SpaceChar}{\sqcup}



\begin{document}

\pagestyle{empty}

%first figure
\begin{picture}(60,50)(0,50)


%\put(-20,-20){\framebox(80,40){}} 
  
%  \gasset{ExtNL=y,NLdist=6,AHnb=0,ilength=-6} 
%% ilength is the length of the tick used to label a node as "initial
%% node"
%% NLdist is the distance between the node label and the node
%% ExtNL=y/n means external to the node yes or no
\gasset{Nh=4,Nw=4}
%\node[Nh=30,Nw=30,Nmr=45](D)(0,0){} 
%\node[fillcolor=Gray](s)(0,0){}
\node[fillcolor=White](a)(20,0){}
%\nodelabel[NLangle=180,ExtNL=y,NLdist=1,ilength=0](s){s}
\nodelabel[NLangle=180,ExtNL=n,NLdist=0,ilength=0](a){\textcolor{Black}{$a$}}

\node[fillcolor=White](f)(40,0){}
%\nodelabel[NLangle=90,ExtNL=y,NLdist=1,ilength=0](x){x}
\nodelabel[NLangle=180,ExtNL=n,NLdist=0,ilength=0](f){\textcolor{Black}{$f$}}



\node[fillcolor=White](b)(15,-10){}
%\nodelabel[NLangle=90,ExtNL=y,NLdist=1,ilength=0](t){t}
\nodelabel[NLangle=180,ExtNL=n,NLdist=0,ilength=0](b){\textcolor{Black}{$b$}}

\node[fillcolor=White](e)(50,-10){}
%\nodelabel[NLangle=-90,ExtNL=y,NLdist=1,ilength=0](z){z}
\nodelabel[NLangle=180,ExtNL=n,NLdist=0,ilength=0](e){\textcolor{Black}{$e$}}


\node[fillcolor=White](c)(20,-20){}
%\nodelabel[NLangle=90,ExtNL=y,NLdist=1,ilength=0](x){x}
\nodelabel[NLangle=180,ExtNL=n,NLdist=0,ilength=0](c){\textcolor{Black}{$c$}}


\node[fillcolor=White](d)(50,-20){}
%\nodelabel[NLangle=-90,ExtNL=y,NLdist=1,ilength=0](y){y}
\nodelabel[NLangle=180,ExtNL=n,NLdist=0,ilength=0](d){\textcolor{Black}{$d$}}




%\node[fillcolor=White](u)(60,0){}
%\nodelabel[NLangle=-90,ExtNL=y,NLdist=1,ilength=0](z){z}
%\nodelabel[NLangle=180,ExtNL=n,NLdist=0,ilength=0](u){\textcolor{Black}{$x_6$}}


%% AHnb stands for Arrow Head number, set it to zero
%% for no arrows
 % \drawedge[AHnb=0](18,22){}
%\gasset{AHlength=3,AHLength=3}
\gasset{AHnb=0}
%  \drawedge(s,t){}
% \drawedge(s,y){}
% % % \drawedge[ELside=r,ELpos=40,linecolor=Red](z,s){\tiny{5}}
%  \drawedge(t,x){}
%  \drawedge(y,x){}
% % \drawedge(y,z){}

%  %\drawedge[curvedepth=-4,ELside=r,linecolor=Red](x,z){\tiny{4}}

% % \drawedge[curvedepth=0,ELside=r](z,u){}
% %\drawedge[curvedepth=0,ELside=r](x,u){}


% % \drawedge[curvedepth=-4,ELside=r](y,t){}
%  \drawedge[curvedepth=0,ELside=r](t,y){}

\drawedge[curvedepth=0,ELside=r](a,f){}
\drawedge[curvedepth=0,ELside=r](a,b){}
\drawedge[curvedepth=0,ELside=r](a,c){}
\drawedge[curvedepth=0,ELside=r](a,e){}
\drawedge[curvedepth=0,ELside=r](b,e){}
\drawedge[curvedepth=0,ELside=r](b,f){}
\drawedge[curvedepth=0,ELside=r](b,d){}
\drawedge[curvedepth=0,ELside=r](d,c){}
\drawedge[curvedepth=0,ELside=r](f,e){}
\drawedge[curvedepth=0,ELside=r](d,e){}

\end{picture}



\end{document}
%%% Local Variables: 
%%% mode: latex
%%% TeX-master: t
%%% End: 
