
\documentclass{article}
\usepackage{gastex}
\usepackage[usenames,dvips]{color}
\usepackage{amsmath}
\usepackage{amssymb}
\usepackage{times}
%\newcommand{\SpaceChar}{\ensuremath{\text{\textvisiblespace}}}
\newcommand{\SpaceChar}{\sqcup}



\begin{document}

\pagestyle{empty}


\begin{picture}(60,50)(40,-60)
 \gasset{Nmr=0}
  \node[Nh=5,Nw=80](A)(30,0){\tiny $(x_1\vee x_2\vee x_3\vee x_4)\wedge \bar{x}_1\wedge (x_1\vee x_2\vee \bar{x}_3)\wedge (x_1\vee \bar{x}_2)\wedge (x_2\vee\bar{x}_4)$}
  \node[Nh=5,Nw=50](B)(15,-20){\tiny $(x_2\vee x_3\vee x_4)\wedge (x_2\vee\bar{x}_3)\wedge \bar{x}_2\wedge (x_2\vee\bar{x}_4)$}
%  \node(C)(50,-20){wait}
%  \node[Nmarks=r](D)(0,-20){critical}

% %\put(-20,-20){\framebox(80,40){}} 
  
% %  \gasset{ExtNL=y,NLdist=6,AHnb=0,ilength=-6} 
% %% ilength is the length of the tick used to label a node as "initial
% %% node"
% %% NLdist is the distance between the node label and the node
% %% ExtNL=y/n means external to the node yes or no

% %\node[Nh=30,Nw=30,Nmr=45](D)(0,0){} 
% %\node[fillcolor=Gray](s)(0,0){}


% \node[fillcolor=White](a)(0,0){}
% \nodelabel[NLangle=180,ExtNL=n,NLdist=0,ilength=0](a){\textcolor{Black}{$a$}}


% \node[fillcolor=White](b)(25,15){}
% \nodelabel[NLangle=180,ExtNL=n,NLdist=0,ilength=0](b){\textcolor{Black}{$b$}}

% \node[fillcolor=White](d)(25,-15){}
% \nodelabel[NLangle=180,ExtNL=n,NLdist=0,ilength=0](d){\textcolor{Black}{$d$}}


% \node[fillcolor=White](c)(60,15){}
% \nodelabel[NLangle=180,ExtNL=n,NLdist=0,ilength=0](c){\textcolor{Black}{$c$}}

% \node[fillcolor=White](e)(50,-15){}
% \nodelabel[NLangle=180,ExtNL=n,NLdist=0,ilength=0](e){\textcolor{Black}{$e$}}

% \node[fillcolor=White](g)(0,-25){}
% \nodelabel[NLangle=180,ExtNL=n,NLdist=0,ilength=0](g){\textcolor{Black}{$g$}}

% %% AHnb stands for Arrow Head number, set it to zero
% %% for no arrows
%  % \drawedge[AHnb=0](18,22){}
% %\gasset{AHlength=3,AHLength=3}
% \gasset{AHnb=0}

%  \drawedge(a,b){}
% \drawedge(a,d){}
% % \drawedge[ELside=r,ELpos=40,linecolor=Red](z,s){\tiny{5}}
% % \drawedge(t,x){}
% % \drawedge(y,x){}
% % \drawedge(y,z){}

%  %\drawedge[curvedepth=-4,ELside=r,linecolor=Red](x,z){\tiny{4}}

%  %\drawedge[curvedepth=0,ELside=r](z,u){}
% %\drawedge[curvedepth=0,ELside=r](x,u){}


% % \drawedge[curvedepth=-4,ELside=r](y,t){}
%  \drawedge[curvedepth=0,ELside=r](b,d){}

%  \drawedge[curvedepth=0,ELside=r](b,c){}
%  \drawedge[curvedepth=0,ELside=r](b,e){}
%  \drawedge[curvedepth=0,ELside=r](d,g){}
%  \drawedge[curvedepth=0,ELside=r](d,e){}

\end{picture}



\end{document}
%%% Local Variables: 
%%% mode: latex
%%% TeX-master: t
%%% End: 
