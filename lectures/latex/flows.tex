\documentclass{beamer}
\usepackage{beamerthemesplit}
\usepackage{graphics}
%\usepackage[lined,boxed]{algorithm2e}
\usepackage[lined,noend]{algorithm2e}
\usepackage{amsmath}
\usepackage{tikz}
\usetikzlibrary{positioning,shapes.multipart}
\usepackage{amssymb}
\usepackage{listings}
\usepackage{soul}
\usepackage{mathtools}
\usepackage{colortbl}
\usepackage{subfigure}
\newcommand{\given}[0]{\ensuremath{\!\mid\!}}
\makeatletter
\newcolumntype{W}{!{\smash{\vrule
\@width 4\arrayrulewidth
\@height\dimexpr\ht\@arstrutbox+2pt\relax
\@depth\dimexpr\dp\@arstrutbox+2pt\relax}}}
\makeatother
\definecolor{gray}{rgb}{.7,.7,.7}



\DeclarePairedDelimiter\ceil{\lceil}{\rceil}
\DeclarePairedDelimiter\floor{\lfloor}{\rfloor}

\lstset{
basicstyle=\small,
keywordstyle=\color{blue}\bfseries,
numbers=left,
numberstyle=\tiny,
numbersep=5pt,
showstringspaces=false,
showspaces=false,
captionpos=b,
frame=tb,
float=tbh,
,escapeinside={*@}{@*}
}
\usetheme{Boadilla}
\title{ Analysis of Algorithms}
\subtitle{Network Flows}
\author{Hikmat Farhat}
%\email{hfarhat@ndu.edu.lb}
%\institution{Notre Dame University}
\newtheorem{mydef}{Definition}
\newtheorem{lem}{Lemma}
%\newcommand{\emphasis}[1]{\textcolor{yellow}{#1}}
%\newcommand{\emphasis}[1]{\hl{#1}}
\newcommand{\emphasis}[1]{\ul{#1}}
%\newcommand{\floor}[1]{\lfloor{#1}\rfloor}
%\newcommand{\bfloor}[1]{\Big\lfloor{#1}\Big\rfloor}

%\newcommand{\gets}[0]{\leftarrow}

%\newcommand{\gets}{\ensuremath{\leftarrow}}
%\DeclareTextFontCommand{\emph}{\emphasis}
\sethlcolor{yellow}
\begin{document}
% title page
\frame{\titlepage}

\begin{frame}
  \frametitle{Maximum Flows}
\begin{itemize}
  \item Imagine having factory that produces materials
  \item You would like to transport your products to a given destination
  \item Suppose that there are multiple roads from factory to destination
  \item Some are congested and some are less some
  \item What is the maximum number of products you could transport from destination to source?
\end{itemize}
  

\end{frame}  
\begin{frame}
  \frametitle{Flow Networks}
\begin{itemize}
  \item A \textbf{flow network} $G=<V,E>$ is a directed graph.
  \item Each edge $(u,v)\in E$ has a \textbf{capacity} $c(u,v)\ge 0$.
  \item If $(u,v)\notin E$ then we set $c(u,v)=0$.
  \item There are two special vertices: \textbf{source} $s\in V$ and \textbf{sink} $t\in V$.
  \item We assume that the graph is connected
  \item A \textbf{flow} is a function $f:V\times V\rightarrow \mathbf{R}$
\end{itemize}
  

\end{frame}

\end{document}


%%% Local Variables:
%%% mode: latex
%%% TeX-master: t
%%% End:
