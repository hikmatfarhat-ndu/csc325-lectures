\documentclass{beamer}
\usepackage{beamerthemesplit}
\usepackage{graphics}
%\usepackage[lined,boxed]{algorithm2e}
\usepackage[lined,noend]{algorithm2e}
\usepackage{amsmath}
\usepackage{tikz}
\usetikzlibrary{positioning,shapes.multipart,automata,arrows.meta}
\usepackage{amssymb}
\usepackage{listings}
\usepackage{soul}
\usepackage{mathtools}
\usepackage{colortbl}
\usepackage{subfigure}
\newcommand{\given}[0]{\ensuremath{\!\mid\!}}
\newcommand{\set}[1]{\ensuremath{\{#1\}}}
\makeatletter
\newcolumntype{W}{!{\smash{\vrule
\@width 4\arrayrulewidth
\@height\dimexpr\ht\@arstrutbox+2pt\relax
\@depth\dimexpr\dp\@arstrutbox+2pt\relax}}}
\makeatother
\definecolor{gray}{rgb}{.7,.7,.7}



\DeclarePairedDelimiter\ceil{\lceil}{\rceil}
\DeclarePairedDelimiter\floor{\lfloor}{\rfloor}

\lstset{
basicstyle=\small,
keywordstyle=\color{blue}\bfseries,
numbers=left,
numberstyle=\tiny,
numbersep=5pt,
showstringspaces=false,
showspaces=false,
captionpos=b,
frame=tb,
float=tbh,
,escapeinside={*@}{@*}
}
\usetheme{Boadilla}
\title{ Analysis of Algorithms}
\subtitle{Network Flows}
\author{Hikmat Farhat}
%\email{hfarhat@ndu.edu.lb}
%\institution{Notre Dame University}
\newtheorem{mydef}{Definition}
\newtheorem{lem}{Lemma}
%\newcommand{\emphasis}[1]{\textcolor{yellow}{#1}}
%\newcommand{\emphasis}[1]{\hl{#1}}
\newcommand{\emphasis}[1]{\ul{#1}}
%\newcommand{\floor}[1]{\lfloor{#1}\rfloor}
%\newcommand{\bfloor}[1]{\Big\lfloor{#1}\Big\rfloor}

%\newcommand{\gets}[0]{\leftarrow}

%\newcommand{\gets}{\ensuremath{\leftarrow}}
%\DeclareTextFontCommand{\emph}{\emphasis}
\sethlcolor{yellow}
\begin{document}
% title page
\frame{\titlepage}

\begin{frame}
  \frametitle{Maximum Flows}
\begin{itemize}
  \item Imagine having factory that produces materials
  \item You would like to transport your products to a given destination
  \item Suppose that there are multiple roads from factory to destination
  \item Some are congested and some are less some
  \item What is the maximum number of products you could transport from destination to source?
\end{itemize}
  

\end{frame}  
\begin{frame}
  \frametitle{Flow Networks}
\begin{itemize}
  \item A \textbf{flow network} $G=<V,E>$ is a directed graph.
  \item Each edge $(u,v)\in E$ has a \textbf{capacity} $c(u,v)\ge 0$.
  \item If $(u,v)\notin E$ then we set $c(u,v)=0$.
  \item There are two special vertices: \textbf{source} $s\in V$ and \textbf{sink} $t\in V$.
  \item We assume that the graph is connected and has no \textbf{anti parallel} edges. If $(u,v)\in E$ implies $(v,u)\notin E$.
  \item A \textbf{flow} is a function $f:V\times V\rightarrow \mathbf{R}$ with the following constraints:
  \begin{enumerate}
    \item for all $u,v\in V$ we have $f(u,v)\le c(u,v)$
    \item for all $u,v\in V$ we have $f(u,v)=-f(v,u)$
    \item for all $u\in V-\set{s,t}$ we have
    \begin{align*}
      \sum_{v\in V}f(u,v)=0
    \end{align*}
  \end{enumerate}
\end{itemize}
  

\end{frame}

\begin{frame}
  \frametitle{Example (All examples are taken from the CLRS book)}
  \begin{itemize}
    \item Notation: $flow/capacity$. if $flow=0$, e.g $v_2\rightarrow v_1$. then just $capacity$
  \end{itemize}
  \begin{figure}[]
    \centering
    
   
  \begin{tikzpicture}

    \node[draw,circle,minimum size=25] (s)  at (-7,1.5) {$s$};
    \node[draw,circle,minimum size=25] (v1) at (-3.5,3) {$v_1$};
    \node[draw,circle,minimum size=25] (v2) at (-3.5,-1) {$v_2$};
    \node[draw,circle,minimum size=25] (v3) at (0,3) {$v_3$};
    \node[draw,circle,minimum size=25] (v4) at (0,-1) {$v_4$};
    \node[draw,circle,minimum size=25] (t) at (3,1.5) {$t$};
    \draw[-{Latex[length=2mm]}](s) edge node[midway,above]{\scriptsize 11/16} (v1);
    \draw[-{Latex[length=2mm]}]  (v1) edge node[midway,above]{\scriptsize 12/12} (v3);
    \draw[-{Latex[length=2mm]}]  (s) edge node[midway,below left]{\scriptsize 8/13}(v2);
    %\draw[-{Latex[length=2mm]}]  (v2) edge [bend left=10]
    % node[midway,left]{\scriptsize{10}} (v1);
    \draw[-{Latex[length=2mm]}]  (v2) edge node[midway,below ]{\scriptsize 11/14} (v4);
    \draw[-{Latex[length=2mm]}]  (v4) edge node[midway,right ]{\scriptsize 7/7} (v3);
    \draw[-{Latex[length=2mm]}]  (v4) edge node[midway,below right ]{\scriptsize 4/4} (t);
    \draw[-{Latex[length=2mm]}]  (v3) edge node[midway,above right ]{\scriptsize 19/20} (t);
    \draw[-{Latex[length=2mm]}]  (v3) edge node[midway,below right]{\scriptsize 4/9} (v2);
    \draw[-{Latex[length=2mm]}]  (v2) edge %[bend left=10]
    node[midway,right]{\scriptsize{1/4}} (v1);
    \end{tikzpicture}
    \caption{1}
    \label{}
  \end{figure}

\end{frame}
\section{Residual Networks}
\begin{frame}
  \frametitle{Residual Networks}

  \begin{itemize}
    \item Given a graph $G=<V,E>$, a flow $f$ in $G$ and a capacity function $c$.
    \item Define the residual capacity of an edge $(u,v)$ as
    \begin{align*}
      c_R(u,v)=c(u,v)-f(u,v)
    \end{align*} 
    \item Intuitivally, the residual capacity of an edge is how much more flow can pass through it.
    \item Note that every $(u,v)\notin E$ also has a residual capacity.
    \item Since the capacity of such pairs is by definition zero then their residual capacity is 
    \begin{align*}
      \forall (u,v)\notin E\ \ c_R(u,v)=-f(u,v)=f(v,u)
    \end{align*} 
  \end{itemize}

\end{frame}

\begin{frame}
  \frametitle{Example residual network}

  \begin{figure}[]
    \centering
    
   \subfigure[Network]
   {
  \begin{tikzpicture}[scale=0.5,every node/.style={scale=0.5,minimum size=22}]

    \node[draw,circle] (s)  at (-10,1.5) {$s$};
    \node[draw,circle] (v1) at (-5.5,3) {$v_1$};
    \node[draw,circle] (v2) at (-5.5,-2) {$v_2$};
    \node[draw,circle] (v3) at (0,3) {$v_3$};
    \node[draw,circle] (v4) at (0,-2) {$v_4$};
    \node[draw,circle] (t) at (3,1.5) {$t$};
    \draw[-{Latex[length=2mm]}](s) edge node[midway,above]{\scriptsize 11/16} (v1);
    \draw[-{Latex[length=2mm]}]  (v1) edge node[midway,above]{\scriptsize 12/12} (v3);
    \draw[-{Latex[length=2mm]}]  (s) edge node[midway,below left]{\scriptsize 8/13}(v2);
    %\draw[-{Latex[length=2mm]}]  (v2) edge [bend left=10]
    % node[midway,left]{\scriptsize{10}} (v1);
    \draw[-{Latex[length=2mm]}]  (v2) edge node[midway,below ]{\scriptsize 11/14} (v4);
    \draw[-{Latex[length=2mm]}]  (v4) edge node[midway,right ]{\scriptsize 7/7} (v3);
    \draw[-{Latex[length=2mm]}]  (v4) edge node[midway,below right ]{\scriptsize 4/4} (t);
    \draw[-{Latex[length=2mm]}]  (v3) edge node[midway,above right ]{\scriptsize 19/20} (t);
    \draw[-{Latex[length=2mm]}]  (v3) edge node[midway,below right]{\scriptsize 4/9} (v2);
    \draw[-{Latex[length=2mm]}]  (v2) edge %[bend left=10]
    node[midway,right]{\scriptsize{1/4}} (v1);
    \end{tikzpicture}
   
   }

   \subfigure[Residual]{

    \begin{tikzpicture}[scale=0.5,every node/.style={scale=0.5,minimum size=22}]

      \node[draw,circle] (s)  at (-10,1.5) { $s$};
      \node[draw,circle] (v1) at (-5.5,3) { $v_1$};
      \node[draw,circle] (v2) at (-5.5,-2) {$v_2$};
      \node[draw,circle] (v3) at (0,3) {$v_3$};
      \node[draw,circle] (v4) at (0,-2) {$v_4$};
      \node[draw,circle] (t) at (3,1.5) {$t$};
      \draw[-{Latex[length=2mm]}](s) edge  [bend left=10 ] 
      node[midway,above]{\scriptsize 5} (v1);
      \draw[-{Latex[length=2mm]}](v1) edge  [bend left=10 ] 
      node[midway,right]{\scriptsize 11} (s);

      \draw[-{Latex[length=2mm]}]  (v3) edge node[midway,above]{\scriptsize 12} (v1);
      \draw[-{Latex[length=2mm]}]  (s) edge [bend left=10] 
      node[midway,right]{\scriptsize 5}(v2);
      \draw[-{Latex[length=2mm]}]  (v2) edge [bend left=10]
      node[midway,left]{\scriptsize 8}(s);

      \draw[-{Latex[length=2mm]}]  (v2) edge [bend left=10]
       node[midway,left]{\scriptsize{3}} (v1);
      \draw[-{Latex[length=2mm]}]  (v2) edge [bend left=10]
      node[midway,above]{\scriptsize 3} (v4);
      \draw[-{Latex[length=2mm]}]  (v4) edge [bend left=10]
      node[midway,below]{\scriptsize 11} (v2);

      \draw[-{Latex[length=2mm]}]  (v3) edge node[midway,right ]{\scriptsize 7} (v4);

      \draw[-{Latex[length=2mm]}]  (t) edge node[midway,below right ]{\scriptsize 4} (v4);
      \draw[-{Latex[length=2mm]}]  (v3) edge [bend left=10]
      node[midway,above]{\scriptsize 1} (t);
      \draw[-{Latex[length=2mm]}]  (t) edge [bend left=10]
      node[midway,below ]{\scriptsize 19} (v3);
      \draw[-{Latex[length=2mm]}]  (v3) edge [bend right=10]
      node[midway,above left]{\scriptsize 5} (v2);
      \draw[-{Latex[length=2mm]}]  (v2) edge [bend right=10]
      node[midway,below]{\scriptsize 4} (v3);

      \draw[-{Latex[length=2mm]}]  (v1) edge [bend left=10]
      node[midway,right]{\scriptsize{1}} (v2);
      \end{tikzpicture}
    

   }
   %\caption{1}
   %\label{}
  \end{figure}

\end{frame}
\begin{frame}
  \frametitle{Ford-Fulkerson Method}

  \begin{itemize}
    \item Ford-Fulkerson is a general \textbf{method} to find a maximum flow in a a network.
    \item Iteratively find an \textbf{augemting path} in the residual network.
    \item update the residual network until there is no more augemting paths.
    \item the resulting flow is maximum.
    \item \textbf{Does not} specify how to find an augmenting path.
    \item For now we will find an augmenting path "visually".

  \end{itemize}

\end{frame}
\begin{frame}[fragile]
  \frametitle{}

  \begin{algorithm}[H]
    \SetKwFunction{KwFn}{FORD-FULKERSON(G,s,t)}
    \DontPrintSemicolon
    %\dontprintsemicolon
    \KwFn \;
   \ForEach{ $(u,v)\in V\times V$}{
       $c_r(u,v)\gets c(u,v)$
   }

   \While {$\exists$ a path $p$ from $s$ to $t$ in $G_f$}
    {
      $c_r(p)\gets \min\set{c_r(u,v): (u,v)\in p} $\;

      \ForEach{$(u,v)\in p$}{
        $c_r(u,v)\gets c_r(u,v)-c_r(p)$\;
        $c_r(v,u)\gets c_r(v,u)+c_r(p)$\;
      }

    }
    \ForEach{ $(u,v)\in E$}{
      $f(u,v)\gets c(u,v)-c_r(u,v)$
    } 
    \end{algorithm}

\end{frame}
\setcounter{subfigure}{0}
\begin{frame}
  \frametitle{Example}
  \begin{itemize}
    \item Initially there is no flow. Only edges with $c_r>0$ are shown
  \end{itemize}
  \begin{figure}[]
    \centering
    
   \subfigure[\tiny{$c_r(p)=4$}]
   {
  \begin{tikzpicture}[scale=0.4,every node/.style={scale=0.5,minimum size=22}]

    \node[draw,circle] (s)  at (-10,1.5) {$s$};
    \node[draw,circle] (v1) at (-5.5,3) {$v_1$};
    \node[draw,circle] (v2) at (-5.5,-2) {$v_2$};
    \node[draw,circle] (v3) at (0,3) {$v_3$};
    \node[draw,circle] (v4) at (0,-2) {$v_4$};
    \node[draw,circle] (t) at (4,1.5) {$t$};
    \draw[-{Latex[length=2mm]}](s) edge[color=red] node[midway,above]{\scriptsize 16} (v1);
    \draw[-{Latex[length=2mm]}]  (v1) edge[color=red] node[midway,above]{\scriptsize 12} (v3);
    \draw[-{Latex[length=2mm]}]  (s) edge node[midway,below left]{\scriptsize 13}(v2);
    
    \draw[-{Latex[length=2mm]}]  (v2) edge [color=red] node[midway,below ]{\scriptsize 14} (v4);
    \draw[-{Latex[length=2mm]}]  (v4) edge node[midway,right ]{\scriptsize 7} (v3);
    \draw[-{Latex[length=2mm]}]  (v4) edge[color=red] node[midway,below right ]{\scriptsize 4} (t);
    \draw[-{Latex[length=2mm]}]  (v3) edge node[midway,above right ]{\scriptsize 20} (t);
    \draw[-{Latex[length=2mm]}]  (v3) edge[color=red] node[midway,below right]{\scriptsize 9} (v2);
    \draw[-{Latex[length=2mm]}]  (v2) edge
    node[midway,left]{\scriptsize{4}} (v1);
    \end{tikzpicture}
   
   }

   \subfigure[\tiny{update 1}]{

    \begin{tikzpicture}[scale=0.4,every node/.style={scale=0.5,minimum size=22}]

      \node[draw,circle] (s)  at (-10,1.5) { $s$};
      \node[draw,circle] (v1) at (-5.5,3) { $v_1$};
      \node[draw,circle] (v2) at (-5.5,-2) {$v_2$};
      \node[draw,circle] (v3) at (0,3) {$v_3$};
      \node[draw,circle] (v4) at (0,-2) {$v_4$};
      \node[draw,circle] (t) at (4,1.5) {$t$};
      \draw[-{Latex[length=2mm]}](s) edge  [bend left=10 ] 
      node[midway,above]{\scriptsize 12} (v1);
      \draw[-{Latex[length=2mm]}](v1) edge  [bend left=10 ] 
      node[midway,right]{\scriptsize 4} (s);

      \draw[-{Latex[length=2mm]}]  (v3) edge[bend left=10] 
      node[midway,below]{\scriptsize 4} (v1);
      \draw[-{Latex[length=2mm]}]  (v1) edge[bend left=10] 
      node[midway,above]{\scriptsize 8} (v3);

      \draw[-{Latex[length=2mm]}]  (s) edge  
      node[midway,right]{\scriptsize 13}(v2);
      %\draw[-{Latex[length=2mm]}]  (v2) edge [bend left=10]
      %#node[midway,left]{\scriptsize 8}(s);

      
      \draw[-{Latex[length=2mm]}]  (v2) edge 
      node[midway,left]{\scriptsize{4}} (v1);

      \draw[-{Latex[length=2mm]}]  (v2) edge [bend left=10]
      node[midway,above]{\scriptsize 10} (v4);
      \draw[-{Latex[length=2mm]}]  (v4) edge [bend left=10]
      node[midway,below]{\scriptsize 4} (v2);

      \draw[-{Latex[length=2mm]}]  (v4) edge node[midway,right ]{\scriptsize 7} (v3);

      \draw[-{Latex[length=2mm]}]  (t) edge node[midway,below right ]{\scriptsize 4} (v4);
      \draw[-{Latex[length=2mm]}]  (v3) edge 
      node[midway,above]{\scriptsize 20} (t);
     % \draw[-{Latex[length=2mm]}]  (t) edge [bend left=10]
     %s node[midway,below ]{\scriptsize 15} (v3);
      \draw[-{Latex[length=2mm]}]  (v3) edge [bend right=10]
      node[midway,above left]{\scriptsize 5} (v2);
      \draw[-{Latex[length=2mm]}]  (v2) edge [bend right=10]
      node[midway,below]{\scriptsize 4} (v3);

      \end{tikzpicture}
    

   }
   %\caption{1}
   %\label{}
  \end{figure}
\end{frame}

\begin{frame}
  \frametitle{Example}
  \begin{figure}[]
    \centering
    
   \subfigure[\tiny{$c_r(p)=4$}]{

    \begin{tikzpicture}[scale=0.4,every node/.style={scale=0.5,minimum size=22}]

      \node[draw,circle] (s)  at (-10,1.5) { $s$};
      \node[draw,circle] (v1) at (-5.5,3) { $v_1$};
      \node[draw,circle] (v2) at (-5.5,-2) {$v_2$};
      \node[draw,circle] (v3) at (0,3) {$v_3$};
      \node[draw,circle] (v4) at (0,-2) {$v_4$};
      \node[draw,circle] (t) at (4,1.5) {$t$};
      \draw[-{Latex[length=2mm]}](s) edge  [bend left=10 ] 
      node[midway,above]{\scriptsize 12} (v1);
      \draw[-{Latex[length=2mm]}](v1) edge  [bend left=10 ] 
      node[midway,right]{\scriptsize 4} (s);

      \draw[-{Latex[length=2mm]}]  (v3) edge[bend left=10] 
      node[midway,below]{\scriptsize 4} (v1);
      \draw[-{Latex[length=2mm]}]  (v1) edge[bend left=10,color=red] 
      node[midway,above]{\scriptsize 8} (v3);

      \draw[-{Latex[length=2mm]}]  (s) edge [color=red] 
      node[midway,right]{\scriptsize 13}(v2);
      %\draw[-{Latex[length=2mm]}]  (v2) edge [bend left=10]
      %#node[midway,left]{\scriptsize 8}(s);

      \draw[-{Latex[length=2mm]}]  (v2) edge [color=red]
       node[midway,left]{\scriptsize{4}} (v1); 
      
      \draw[-{Latex[length=2mm]}]  (v2) edge [bend left=10]
      node[midway,above]{\scriptsize 10} (v4);
      \draw[-{Latex[length=2mm]}]  (v4) edge [bend left=10]
      node[midway,below]{\scriptsize 4} (v2);

      \draw[-{Latex[length=2mm]}]  (v4) edge node[midway,right ]{\scriptsize 7} (v3);

      \draw[-{Latex[length=2mm]}]  (t) edge node[midway,below right ]{\scriptsize 4} (v4);
      \draw[-{Latex[length=2mm]}]  (v3) edge [color=red]
      node[midway,above]{\scriptsize 20} (t);
     % \draw[-{Latex[length=2mm]}]  (t) edge [bend left=10]
     %s node[midway,below ]{\scriptsize 15} (v3);
      \draw[-{Latex[length=2mm]}]  (v3) edge [bend right=10]
      node[midway,above left]{\scriptsize 5} (v2);
      \draw[-{Latex[length=2mm]}]  (v2) edge [bend right=10]
      node[midway,below]{\scriptsize 4} (v3);

      \end{tikzpicture}
    

   }

   \subfigure[\tiny{update 2}]{

    \begin{tikzpicture}[scale=0.4,every node/.style={scale=0.5,minimum size=22}]

      \node[draw,circle] (s)  at (-10,1.5) { $s$};
      \node[draw,circle] (v1) at (-5.5,3) { $v_1$};
      \node[draw,circle] (v2) at (-5.5,-2) {$v_2$};
      \node[draw,circle] (v3) at (0,3) {$v_3$};
      \node[draw,circle] (v4) at (0,-2) {$v_4$};
      \node[draw,circle] (t) at (4,1.5) {$t$};
      \draw[-{Latex[length=2mm]}](s) edge  [bend left=10 ] 
      node[midway,above]{\scriptsize 12} (v1);
      \draw[-{Latex[length=2mm]}](v1) edge  [bend left=10 ] 
      node[midway,right]{\scriptsize 4} (s);

      \draw[-{Latex[length=2mm]}]  (v3) edge[bend left=10] 
      node[midway,below]{\scriptsize 8} (v1);
      \draw[-{Latex[length=2mm]}]  (v1) edge[bend left=10] 
      node[midway,above]{\scriptsize 4} (v3);

      \draw[-{Latex[length=2mm]}]  (s) edge  [bend left=10]
      node[midway,right]{\scriptsize 9}(v2);
      \draw[-{Latex[length=2mm]}]  (v2) edge [bend left=10]
      node[midway,left]{\scriptsize 4}(s);

      
      \draw[-{Latex[length=2mm]}]  (v1) edge 
      node[midway,right]{\scriptsize{4}} (v2);

      \draw[-{Latex[length=2mm]}]  (v2) edge [bend left=10]
      node[midway,above]{\scriptsize 10} (v4);
      \draw[-{Latex[length=2mm]}]  (v4) edge [bend left=10]
      node[midway,below]{\scriptsize 4} (v2);

      \draw[-{Latex[length=2mm]}]  (v4) edge node[midway,right ]{\scriptsize 7} (v3);

      \draw[-{Latex[length=2mm]}]  (t) edge node[midway,below right ]{\scriptsize 4} (v4);
      
      \draw[-{Latex[length=2mm]}]  (v3) edge [bend left=10]
      node[midway,above]{\scriptsize 16} (t);
      \draw[-{Latex[length=2mm]}]  (t) edge [bend left=10]
      node[midway,below]{\scriptsize 4} (v3);
     % \draw[-{Latex[length=2mm]}]  (t) edge [bend left=10]
     %s node[midway,below ]{\scriptsize 15} (v3);
      \draw[-{Latex[length=2mm]}]  (v3) edge [bend right=10]
      node[midway,above left]{\scriptsize 5} (v2);
      \draw[-{Latex[length=2mm]}]  (v2) edge [bend right=10]
      node[midway,below]{\scriptsize 4} (v3);

      \end{tikzpicture}
   }
   %\caption{1}
   %\label{}
  \end{figure}
\end{frame}


\begin{frame}
  \frametitle{Example}
 
  \begin{figure}[]
    \centering
   \subfigure[\tiny{$c_r(p)=4$}]{

    \begin{tikzpicture}[scale=0.4,every node/.style={scale=0.5,minimum size=22}]

      \node[draw,circle] (s)  at (-10,1.5) { $s$};
      \node[draw,circle] (v1) at (-5.5,3) { $v_1$};
      \node[draw,circle] (v2) at (-5.5,-2) {$v_2$};
      \node[draw,circle] (v3) at (0,3) {$v_3$};
      \node[draw,circle] (v4) at (0,-2) {$v_4$};
      \node[draw,circle] (t) at (4,1.5) {$t$};
      \draw[-{Latex[length=2mm]}](s) edge  [bend left=10,color=red ] 
      node[midway,above]{\scriptsize 12} (v1);
      \draw[-{Latex[length=2mm]}](v1) edge  [bend left=10 ] 
      node[midway,right]{\scriptsize 4} (s);

      \draw[-{Latex[length=2mm]}]  (v3) edge[bend left=10] 
      node[midway,below]{\scriptsize 8} (v1);
      \draw[-{Latex[length=2mm]}]  (v1) edge[bend left=10] 
      node[midway,above]{\scriptsize 4} (v3);

      \draw[-{Latex[length=2mm]}]  (s) edge  [bend left=10]
      node[midway,right]{\scriptsize 9}(v2);
      \draw[-{Latex[length=2mm]}]  (v2) edge [bend left=10]
      node[midway,left]{\scriptsize 4}(s);

      
      \draw[-{Latex[length=2mm]}]  (v1) edge [color=red] 
      node[midway,right]{\scriptsize{4}} (v2);

      \draw[-{Latex[length=2mm]}]  (v2) edge [bend left=10]
      node[midway,above]{\scriptsize 10} (v4);
      \draw[-{Latex[length=2mm]}]  (v4) edge [bend left=10]
      node[midway,below]{\scriptsize 4} (v2);

      \draw[-{Latex[length=2mm]}]  (v4) edge node[midway,right ]{\scriptsize 7} (v3);

      \draw[-{Latex[length=2mm]}]  (t) edge node[midway,below right ]{\scriptsize 4} (v4);
      
      \draw[-{Latex[length=2mm]}]  (v3) edge [bend left=10,color=red]
      node[midway,above]{\scriptsize 16} (t);
      \draw[-{Latex[length=2mm]}]  (t) edge [bend left=10]
      node[midway,below]{\scriptsize 4} (v3);
     % \draw[-{Latex[length=2mm]}]  (t) edge [bend left=10]
     %s node[midway,below ]{\scriptsize 15} (v3);
      \draw[-{Latex[length=2mm]}]  (v3) edge [bend right=10]
      node[midway,above left]{\scriptsize 5} (v2);
      \draw[-{Latex[length=2mm]}]  (v2) edge [bend right=10,color=red]
      node[midway,below]{\scriptsize 4} (v3);

      \end{tikzpicture}
   }
   
   \subfigure[\tiny{update 3}]{

    \begin{tikzpicture}[scale=0.4,every node/.style={scale=0.5,minimum size=22}]

      \node[draw,circle] (s)  at (-10,1.5) { $s$};
      \node[draw,circle] (v1) at (-5.5,3) { $v_1$};
      \node[draw,circle] (v2) at (-5.5,-2) {$v_2$};
      \node[draw,circle] (v3) at (0,3) {$v_3$};
      \node[draw,circle] (v4) at (0,-2) {$v_4$};
      \node[draw,circle] (t) at (4,1.5) {$t$};
      \draw[-{Latex[length=2mm]}](s) edge  [bend left=10 ] 
      node[midway,above]{\scriptsize 8} (v1);
      \draw[-{Latex[length=2mm]}](v1) edge  [bend left=10 ] 
      node[midway,right]{\scriptsize 8} (s);

      \draw[-{Latex[length=2mm]}]  (v3) edge [bend left=10]
      node[midway,below]{\scriptsize 8} (v1);
      \draw[-{Latex[length=2mm]}]  (v1) edge [bend left=10]
      node[midway,above]{\scriptsize 4} (v3);
     % \draw[-{Latex[length=2mm]}]  (v1) edge[bend left=10] 
     % node[midway,above]{\scriptsize 8} (v3);

      \draw[-{Latex[length=2mm]}]  (s) edge  [bend left=10]
      node[midway,right]{\scriptsize 9}(v2);
      \draw[-{Latex[length=2mm]}]  (v2) edge  [bend left=10]
      node[midway,below]{\scriptsize 4}(s);
      %\draw[-{Latex[length=2mm]}]  (v2) edge [bend left=10]
      %#node[midway,left]{\scriptsize 8}(s);

      \draw[-{Latex[length=2mm]}]  (v2) edge 
       node[midway,left]{\scriptsize{4}} (v1); 
      %\draw[-{Latex[length=2mm]}]  (v1) edge [bend left=10]
      %node[midway,right]{\scriptsize{11}} (v2);

      \draw[-{Latex[length=2mm]}]  (v2) edge [bend left=10]
      node[midway,above]{\scriptsize 10} (v4);
      \draw[-{Latex[length=2mm]}]  (v4) edge [bend left=10]
      node[midway,below]{\scriptsize 4} (v2);

      \draw[-{Latex[length=2mm]}]  (v4) edge node[midway,right ]{\scriptsize 7} (v3);

      \draw[-{Latex[length=2mm]}]  (t) edge node[midway,below right ]{\scriptsize 4} (v4);
      
      \draw[-{Latex[length=2mm]}]  (v3) edge [bend left=10]
      node[midway,above]{\scriptsize 12} (t);
      \draw[-{Latex[length=2mm]}]  (t) edge [bend left=10]
      node[midway,below]{\scriptsize 8} (v3);
     % \draw[-{Latex[length=2mm]}]  (t) edge [bend left=10]
     %s node[midway,below ]{\scriptsize 15} (v3);
      \draw[-{Latex[length=2mm]}]  (v3) edge 
      node[midway,above left]{\scriptsize 9} (v2);
      %\draw[-{Latex[length=2mm]}]  (v2) edge [bend right=10]
      %node[midway,below]{\scriptsize 4} (v3);

      \end{tikzpicture}
   }
   %\caption{1}
   %\label{}
  \end{figure}
\end{frame}

\begin{frame}
  \frametitle{Example}
 
  \begin{figure}[]
    \centering
  
    \subfigure[\tiny{$c_r(p)=7$}]{

    \begin{tikzpicture}[scale=0.4,every node/.style={scale=0.5,minimum size=22}]

      \node[draw,circle] (s)  at (-10,1.5) { $s$};
      \node[draw,circle] (v1) at (-5.5,3) { $v_1$};
      \node[draw,circle] (v2) at (-5.5,-2) {$v_2$};
      \node[draw,circle] (v3) at (0,3) {$v_3$};
      \node[draw,circle] (v4) at (0,-2) {$v_4$};
      \node[draw,circle] (t) at (4,1.5) {$t$};
      \draw[-{Latex[length=2mm]}](s) edge  [bend left=10 ] 
      node[midway,above]{\scriptsize 8} (v1);
      \draw[-{Latex[length=2mm]}](v1) edge  [bend left=10 ] 
      node[midway,right]{\scriptsize 8} (s);

      \draw[-{Latex[length=2mm]}]  (v3) edge [bend left=10]
      node[midway,below]{\scriptsize 8} (v1);
      \draw[-{Latex[length=2mm]}]  (v1) edge [bend left=10]
      node[midway,above]{\scriptsize 4} (v3);
     % \draw[-{Latex[length=2mm]}]  (v1) edge[bend left=10] 
     % node[midway,above]{\scriptsize 8} (v3);

      \draw[-{Latex[length=2mm]}]  (s) edge  [bend left=10,color=red]
      node[midway,right]{\scriptsize 9}(v2);
      \draw[-{Latex[length=2mm]}]  (v2) edge  [bend left=10]
      node[midway,below]{\scriptsize 4}(s);
      %\draw[-{Latex[length=2mm]}]  (v2) edge [bend left=10]
      %#node[midway,left]{\scriptsize 8}(s);

      \draw[-{Latex[length=2mm]}]  (v2) edge 
       node[midway,left]{\scriptsize{4}} (v1); 
      %\draw[-{Latex[length=2mm]}]  (v1) edge [bend left=10]
      %node[midway,right]{\scriptsize{11}} (v2);

      \draw[-{Latex[length=2mm]}]  (v2) edge [bend left=10,color=red]
      node[midway,above]{\scriptsize 10} (v4);
      \draw[-{Latex[length=2mm]}]  (v4) edge [bend left=10]
      node[midway,below]{\scriptsize 4} (v2);

      \draw[-{Latex[length=2mm]}]  (v4) edge[color=red] node[midway,right ]{\scriptsize 7} (v3);

      \draw[-{Latex[length=2mm]}]  (t) edge node[midway,below right ]{\scriptsize 4} (v4);
      
      \draw[-{Latex[length=2mm]}]  (v3) edge [bend left=10,color=red]
      node[midway,above]{\scriptsize 12} (t);
      \draw[-{Latex[length=2mm]}]  (t) edge [bend left=10]
      node[midway,below]{\scriptsize 8} (v3);
     % \draw[-{Latex[length=2mm]}]  (t) edge [bend left=10]
     %s node[midway,below ]{\scriptsize 15} (v3);
      \draw[-{Latex[length=2mm]}]  (v3) edge 
      node[midway,above left]{\scriptsize 9} (v2);
      %\draw[-{Latex[length=2mm]}]  (v2) edge [bend right=10]
      %node[midway,below]{\scriptsize 4} (v3);

      \end{tikzpicture}
   }
   \subfigure[\tiny{update 4}]{

    \begin{tikzpicture}[scale=0.4,every node/.style={scale=0.5,minimum size=22}]

      \node[draw,circle] (s)  at (-10,1.5) { $s$};
      \node[draw,circle] (v1) at (-5.5,3) { $v_1$};
      \node[draw,circle] (v2) at (-5.5,-2) {$v_2$};
      \node[draw,circle] (v3) at (0,3) {$v_3$};
      \node[draw,circle] (v4) at (0,-2) {$v_4$};
      \node[draw,circle] (t) at (4,1.5) {$t$};
      \draw[-{Latex[length=2mm]}](s) edge  [bend left=10 ] 
      node[midway,above]{\scriptsize 8} (v1);
      \draw[-{Latex[length=2mm]}](v1) edge  [bend left=10 ] 
      node[midway,right]{\scriptsize 8} (s);

      \draw[-{Latex[length=2mm]}]  (v3) edge [bend left=10]
      node[midway,below]{\scriptsize 8} (v1);
      \draw[-{Latex[length=2mm]}]  (v1) edge [bend left=10]
      node[midway,above]{\scriptsize 4} (v3);
     % \draw[-{Latex[length=2mm]}]  (v1) edge[bend left=10] 
     % node[midway,above]{\scriptsize 8} (v3);

      \draw[-{Latex[length=2mm]}]  (s) edge  [bend left=10]
      node[midway,right]{\scriptsize 2}(v2);
      \draw[-{Latex[length=2mm]}]  (v2) edge  [bend left=10]
      node[midway,below]{\scriptsize 11}(s);
      %\draw[-{Latex[length=2mm]}]  (v2) edge [bend left=10]
      %#node[midway,left]{\scriptsize 8}(s);

      \draw[-{Latex[length=2mm]}]  (v2) edge 
       node[midway,left]{\scriptsize{4}} (v1); 
      %\draw[-{Latex[length=2mm]}]  (v1) edge [bend left=10]
      %node[midway,right]{\scriptsize{11}} (v2);

      \draw[-{Latex[length=2mm]}]  (v2) edge [bend left=10]
      node[midway,above]{\scriptsize 3} (v4);
      \draw[-{Latex[length=2mm]}]  (v4) edge [bend left=10]
      node[midway,below]{\scriptsize 11} (v2);

      \draw[-{Latex[length=2mm]}]  (v3) edge node[midway,right ]{\scriptsize 7} (v4);

      \draw[-{Latex[length=2mm]}]  (t) edge node[midway,below right ]{\scriptsize 4} (v4);
      
      \draw[-{Latex[length=2mm]}]  (v3) edge [bend left=10]
      node[midway,above]{\scriptsize 5} (t);
      \draw[-{Latex[length=2mm]}]  (t) edge [bend left=10]
      node[midway,below]{\scriptsize 15} (v3);
     % \draw[-{Latex[length=2mm]}]  (t) edge [bend left=10]
     %s node[midway,below ]{\scriptsize 15} (v3);
      \draw[-{Latex[length=2mm]}]  (v3) edge 
      node[midway,above left]{\scriptsize 9} (v2);
      %\draw[-{Latex[length=2mm]}]  (v2) edge [bend right=10]
      %node[midway,below]{\scriptsize 4} (v3);

      \end{tikzpicture}
   }
   %\caption{1}
   %\label{}
  \end{figure}
\end{frame}

\begin{frame}
  \frametitle{Example}
 
  \begin{figure}[]
    \centering
  
 
   \subfigure[\tiny{$c_r(p)=4$}]{

    \begin{tikzpicture}[scale=0.4,every node/.style={scale=0.5,minimum size=22}]

      \node[draw,circle] (s)  at (-10,1.5) { $s$};
      \node[draw,circle] (v1) at (-5.5,3) { $v_1$};
      \node[draw,circle] (v2) at (-5.5,-2) {$v_2$};
      \node[draw,circle] (v3) at (0,3) {$v_3$};
      \node[draw,circle] (v4) at (0,-2) {$v_4$};
      \node[draw,circle] (t) at (4,1.5) {$t$};
      \draw[-{Latex[length=2mm]}](s) edge  [bend left=10 ,color=red] 
      node[midway,above]{\scriptsize 8} (v1);
      \draw[-{Latex[length=2mm]}](v1) edge  [bend left=10 ] 
      node[midway,right]{\scriptsize 8} (s);

      \draw[-{Latex[length=2mm]}]  (v3) edge [bend left=10]
      node[midway,below]{\scriptsize 8} (v1);
      \draw[-{Latex[length=2mm]}]  (v1) edge [bend left=10,color=red]
      node[midway,above]{\scriptsize 4} (v3);
     % \draw[-{Latex[length=2mm]}]  (v1) edge[bend left=10] 
     % node[midway,above]{\scriptsize 8} (v3);

      \draw[-{Latex[length=2mm]}]  (s) edge  [bend left=10]
      node[midway,right]{\scriptsize 2}(v2);
      \draw[-{Latex[length=2mm]}]  (v2) edge  [bend left=10]
      node[midway,below]{\scriptsize 11}(s);
      %\draw[-{Latex[length=2mm]}]  (v2) edge [bend left=10]
      %#node[midway,left]{\scriptsize 8}(s);

      \draw[-{Latex[length=2mm]}]  (v2) edge 
       node[midway,left]{\scriptsize{4}} (v1); 
      %\draw[-{Latex[length=2mm]}]  (v1) edge [bend left=10]
      %node[midway,right]{\scriptsize{11}} (v2);

      \draw[-{Latex[length=2mm]}]  (v2) edge [bend left=10]
      node[midway,above]{\scriptsize 3} (v4);
      \draw[-{Latex[length=2mm]}]  (v4) edge [bend left=10]
      node[midway,below]{\scriptsize 11} (v2);

      \draw[-{Latex[length=2mm]}]  (v3) edge node[midway,right ]{\scriptsize 7} (v4);

      \draw[-{Latex[length=2mm]}]  (t) edge node[midway,below right ]{\scriptsize 4} (v4);
      
      \draw[-{Latex[length=2mm]}]  (v3) edge [bend left=10,color=red]
      node[midway,above]{\scriptsize 5} (t);
      \draw[-{Latex[length=2mm]}]  (t) edge [bend left=10]
      node[midway,below]{\scriptsize 15} (v3);
     % \draw[-{Latex[length=2mm]}]  (t) edge [bend left=10]
     %s node[midway,below ]{\scriptsize 15} (v3);
      \draw[-{Latex[length=2mm]}]  (v3) edge 
      node[midway,above left]{\scriptsize 9} (v2);
      %\draw[-{Latex[length=2mm]}]  (v2) edge [bend right=10]
      %node[midway,below]{\scriptsize 4} (v3);

      \end{tikzpicture}
   }
   
   \subfigure[\tiny{update 5}]{

    \begin{tikzpicture}[scale=0.4,every node/.style={scale=0.5,minimum size=22}]

      \node[draw,circle] (s)  at (-10,1.5) { $s$};
      \node[draw,circle] (v1) at (-5.5,3) { $v_1$};
      \node[draw,circle] (v2) at (-5.5,-2) {$v_2$};
      \node[draw,circle] (v3) at (0,3) {$v_3$};
      \node[draw,circle] (v4) at (0,-2) {$v_4$};
      \node[draw,circle] (t) at (4,1.5) {$t$};
      \draw[-{Latex[length=2mm]}](s) edge  [bend left=10 ] 
      node[midway,above]{\scriptsize 4} (v1);
      \draw[-{Latex[length=2mm]}](v1) edge  [bend left=10 ] 
      node[midway,right]{\scriptsize 12} (s);

      \draw[-{Latex[length=2mm]}]  (v3) edge 
      node[midway,below]{\scriptsize 12} (v1);
      %\draw[-{Latex[length=2mm]}]  (v1) edge [bend left=10]
      %node[midway,above]{\scriptsize 4} (v3);
     % \draw[-{Latex[length=2mm]}]  (v1) edge[bend left=10] 
     % node[midway,above]{\scriptsize 8} (v3);

      \draw[-{Latex[length=2mm]}]  (s) edge  [bend left=10]
      node[midway,right]{\scriptsize 2}(v2);
      \draw[-{Latex[length=2mm]}]  (v2) edge  [bend left=10]
      node[midway,below]{\scriptsize 11}(s);
      %\draw[-{Latex[length=2mm]}]  (v2) edge [bend left=10]
      %#node[midway,left]{\scriptsize 8}(s);

      \draw[-{Latex[length=2mm]}]  (v2) edge 
       node[midway,left]{\scriptsize{4}} (v1); 
      %\draw[-{Latex[length=2mm]}]  (v1) edge [bend left=10]
      %node[midway,right]{\scriptsize{11}} (v2);

      \draw[-{Latex[length=2mm]}]  (v2) edge [bend left=10]
      node[midway,above]{\scriptsize 3} (v4);
      \draw[-{Latex[length=2mm]}]  (v4) edge [bend left=10]
      node[midway,below]{\scriptsize 11} (v2);

      \draw[-{Latex[length=2mm]}]  (v3) edge node[midway,right ]{\scriptsize 7} (v4);

      \draw[-{Latex[length=2mm]}]  (t) edge node[midway,below right ]{\scriptsize 4} (v4);
      
      \draw[-{Latex[length=2mm]}]  (v3) edge [bend left=10]
      node[midway,above]{\scriptsize 1} (t);
      \draw[-{Latex[length=2mm]}]  (t) edge [bend left=10]
      node[midway,below]{\scriptsize 19} (v3);
     % \draw[-{Latex[length=2mm]}]  (t) edge [bend left=10]
     %s node[midway,below ]{\scriptsize 15} (v3);
      \draw[-{Latex[length=2mm]}]  (v3) edge 
      node[midway,above left]{\scriptsize 9} (v2);
      %\draw[-{Latex[length=2mm]}]  (v2) edge [bend right=10]
      %node[midway,below]{\scriptsize 4} (v3);

      \end{tikzpicture}
   }

   %\caption{1}
   %\label{}
  \end{figure}
\end{frame}



\begin{frame}
  \frametitle{Compute the flow}
 \begin{itemize}
  \item for each $(u,v)\in E$ we have $f(u,v)=c_r(v,u)$. Only edges with $c_r>0$ are shown
   %\item for each $(u,v)\in E$ compute $f(u,v)=c(u,v)-c_r(u,v)$.
  % \item $f(s,v_1)=16-4=12; f(v_1,v_3)=12-0=12;f(v_2,v_1)=4-4=0;$
  % \item $f(s,v_2)=13-2=11; f(v_3,v_2)=9-9=0;f(v_2,v_4)=14-3=11;$
   %\item $f(v_3,t)=20-1=19;f(v_4,t)=4-0=4;f(v_4,v_3)=7-0=7;$
 \end{itemize}
  \begin{figure}[]
    \centering
    \subfigure[\tiny{Final Residual Network}]{

      \begin{tikzpicture}[scale=0.4,every node/.style={scale=0.5,minimum size=22}]
  
        \node[draw,circle] (s)  at (-10,1.5) { $s$};
        \node[draw,circle] (v1) at (-5.5,3) { $v_1$};
        \node[draw,circle] (v2) at (-5.5,-2) {$v_2$};
        \node[draw,circle] (v3) at (0,3) {$v_3$};
        \node[draw,circle] (v4) at (0,-2) {$v_4$};
        \node[draw,circle] (t) at (4,1.5) {$t$};
        \draw[-{Latex[length=2mm]}](s) edge  [bend left=10 ] 
        node[midway,above]{\scriptsize 4} (v1);
        \draw[-{Latex[length=2mm]}](v1) edge  [bend left=10 ] 
        node[midway,right]{\scriptsize 12} (s);
  
        \draw[-{Latex[length=2mm]}]  (v3) edge 
        node[midway,below]{\scriptsize 12} (v1);
        %\draw[-{Latex[length=2mm]}]  (v1) edge [bend left=10]
        %node[midway,above]{\scriptsize 4} (v3);
       % \draw[-{Latex[length=2mm]}]  (v1) edge[bend left=10] 
       % node[midway,above]{\scriptsize 8} (v3);
  
        \draw[-{Latex[length=2mm]}]  (s) edge  [bend left=10]
        node[midway,right]{\scriptsize 2}(v2);
        \draw[-{Latex[length=2mm]}]  (v2) edge  [bend left=10]
        node[midway,below]{\scriptsize 11}(s);
        %\draw[-{Latex[length=2mm]}]  (v2) edge [bend left=10]
        %#node[midway,left]{\scriptsize 8}(s);
  
        \draw[-{Latex[length=2mm]}]  (v2) edge 
         node[midway,left]{\scriptsize{4}} (v1); 
        %\draw[-{Latex[length=2mm]}]  (v1) edge [bend left=10]
        %node[midway,right]{\scriptsize{11}} (v2);
  
        \draw[-{Latex[length=2mm]}]  (v2) edge [bend left=10]
        node[midway,above]{\scriptsize 3} (v4);
        \draw[-{Latex[length=2mm]}]  (v4) edge [bend left=10]
        node[midway,below]{\scriptsize 11} (v2);
  
        \draw[-{Latex[length=2mm]}]  (v3) edge node[midway,right ]{\scriptsize 7} (v4);
  
        \draw[-{Latex[length=2mm]}]  (t) edge node[midway,below right ]{\scriptsize 4} (v4);
        
        \draw[-{Latex[length=2mm]}]  (v3) edge [bend left=10]
        node[midway,above]{\scriptsize 1} (t);
        \draw[-{Latex[length=2mm]}]  (t) edge [bend left=10]
        node[midway,below]{\scriptsize 19} (v3);
       % \draw[-{Latex[length=2mm]}]  (t) edge [bend left=10]
       %s node[midway,below ]{\scriptsize 15} (v3);
        \draw[-{Latex[length=2mm]}]  (v3) edge 
        node[midway,above left]{\scriptsize 9} (v2);
        %\draw[-{Latex[length=2mm]}]  (v2) edge [bend right=10]
        %node[midway,below]{\scriptsize 4} (v3);
  
        \end{tikzpicture}
     }
     \subfigure[\tiny{Maximum flow}]{
    \begin{tikzpicture}[scale=0.4,every node/.style={scale=0.5,minimum size=22}]

      \node[draw,circle] (s)  at (-10,1.5) { $s$};
      \node[draw,circle] (v1) at (-5.5,3) { $v_1$};
      \node[draw,circle] (v2) at (-5.5,-2) {$v_2$};
      \node[draw,circle] (v3) at (0,3) {$v_3$};
      \node[draw,circle] (v4) at (0,-2) {$v_4$};
      \node[draw,circle] (t) at (4,1.5) {$t$};
      \draw[-{Latex[length=2mm]}](s) edge  
      node[midway,above]{\scriptsize 12/16} (v1);
      
      \draw[-{Latex[length=2mm]}]  (v1) edge 
      node[midway,above]{\scriptsize 12/12} (v3);
      %\draw[-{Latex[length=2mm]}]  (v1) edge [bend left=10]
      %node[midway,above]{\scriptsize 4} (v3);
     % \draw[-{Latex[length=2mm]}]  (v1) edge[bend left=10] 
     % node[midway,above]{\scriptsize 8} (v3);

      \draw[-{Latex[length=2mm]}]  (s) edge 
      node[midway,below left]{\scriptsize 11/13}(v2);
     
      %\draw[-{Latex[length=2mm]}]  (v2) edge [bend left=10]
      %#node[midway,left]{\scriptsize 8}(s);

      \draw[-{Latex[length=2mm]}]  (v2) edge 
       node[midway,left]{\scriptsize{4}} (v1); 
      %\draw[-{Latex[length=2mm]}]  (v1) edge [bend left=10]
      %node[midway,right]{\scriptsize{11}} (v2);

      \draw[-{Latex[length=2mm]}]  (v2) edge 
      node[midway,above]{\scriptsize 11/14} (v4);
      
      \draw[-{Latex[length=2mm]}]  (v4) edge node[midway,right ]{\scriptsize 7/7} (v3);

      \draw[-{Latex[length=2mm]}]  (v4) edge node[midway,below right ]{\scriptsize 4/4} (t);
      
      \draw[-{Latex[length=2mm]}]  (v3) edge 
      node[midway,above]{\scriptsize 19/20} (t);
      
      \draw[-{Latex[length=2mm]}]  (v3) edge 
      node[midway,above left]{\scriptsize 9} (v2);
      %\draw[-{Latex[length=2mm]}]  (v2) edge [bend right=10]
      %node[midway,below]{\scriptsize 4} (v3);

      \end{tikzpicture}
     }
   %\caption{1}
   %\label{}
  \end{figure}
\end{frame}
\section{Edmonds-Karp Algorithm}
\begin{frame}
  \frametitle{Edmonds-Karp Algorithm}
\begin{itemize}
  \item Uses breadth-first-search (BFS) to find an augmenting path.
  \item We assign a unit weight for each edge and compute the shortest path from $s$ to $t$
  \item Select the shortest path as the augmenting path $p$.
\end{itemize}
  

\end{frame}
\section{Bipartite Matching}
\begin{frame}
  \frametitle{Bipartite matching}

  \begin{itemize}
    \item Given a graph $G=<V,E>$  a \textbf{matching} is a set of edges $M\subseteq E$ such that for all $v\in V$ at \textbf{most} one edge in $M$ is incident on $v$.
    \item $v\in V$ is \textbf{matched} if $\exists (u,v)\in M$ for some $u\in V$.
    \item $M$ is said to be a maximum matching if for all matching $M'$  we have $|M'|\le |M|$
    \item A graph $G=<V,E>$ is said to be bipartite if it can be partitioned $V=L\cup R$ where $L\cap R=\emptyset$ and for all $(u,v)\in E$, $u\in L$ and $v\in R$.
  \end{itemize}

\end{frame}
\begin{frame}
  \frametitle{Example: Matching in bipartite graphs}
  \begin{figure}
    \centering
    
    
  
  \subfigure{
  \tikzstyle{round} = [draw,circle,minimum size=20]
  \begin{tikzpicture}[scale=0.7]
    \node (L) at (-5,8) {$L$};
    \node (R) at (-2,8) {$R$};
    \node[style=round] (v3) at (-5,4.5) {};
    \node[style=round] (v4) at (-5,3) {};
    \node[style=round] (v8) at (-5,1.5) {};
    \node[style=round] (v9) at (-5,-0.5) {};
    \node[style=round] (v1) at (-5,6) {};
    \node[style=round] (v2) at (-2,5.5) {};
    \node[style=round] (v5) at (-2,4) {};
    \node[style=round] (v6) at (-2,2) {};
    \node[style=round] (v7) at (-2,0.5) {};
    \node (label1) at (-3,-1.5) {matching};
    \draw  (v1) edge[ultra thick] (v2);
    \draw  (v3) edge (v2);
    \draw  (v4) edge (v5);
    \draw  (v5) edge (v3);
    \draw  (v3) edge[ultra thick] (v6);
    \draw  (v4) edge (v6);
    \draw  (v4) edge [ultra thick]  (v7);
    \draw  (v8) edge (v6);
    \draw  (v9) edge (v7);
    \draw  (v9) edge (v6);
    \end{tikzpicture}
  }%%
  \subfigure{
  \tikzstyle{round} = [draw,circle,minimum size=20]
  \begin{tikzpicture}[scale=0.7]
    \node (L) at (-5,8) {$L$};
    \node (R) at (-2,8) {$R$};
    \node[style=round] (v3) at (-5,4.5) {};
    \node[style=round] (v4) at (-5,3) {};
    \node[style=round] (v8) at (-5,1.5) {};
    \node[style=round] (v9) at (-5,-0.5) {};
    \node[style=round] (v1) at (-5,6) {};
    \node[style=round] (v2) at (-2,5.5) {};
    \node[style=round] (v5) at (-2,4) {};
    \node[style=round] (v6) at (-2,2) {};
    \node[style=round] (v7) at (-2,0.5) {};
    \node (label1) at (-3,-1.5) {maximum matching};
    \draw  (v1) edge[ultra thick] (v2);
    \draw  (v3) edge (v2);
    \draw  (v4) edge[ultra thick] (v5);
    \draw  (v5) edge (v3);
    \draw  (v3) edge[ultra thick] (v6);
    \draw  (v4) edge (v6);
    \draw  (v4) edge (v7);
    \draw  (v8) edge (v6);
    \draw  (v9) edge[ultra thick] (v7);
    \draw  (v9) edge (v6);
    \end{tikzpicture}
  }
\end{figure}
\end{frame}
\begin{frame}
  \frametitle{Constructing an equivalent flow network}
  \begin{itemize}
    \item Given a bipartite graph $G=<V,E>$ we construct a new (flow) graph $G'=<V',E'>$ as follows:
    \item $V'=V\cup {s,t}$. With $s\in L$ and $t\in R$.
    \item $E'=E\cup\set{(s,u)\mid u\in L)}\cup\set{(u,t)\mid u\in R}$
    \item Also every edge $(u,v)\in E$ is made a direct edge from $L$ to $R$.
    \item Finally the capacity of every edge in $E'$ is set to 1.
  \end{itemize}
\end{frame}
\begin{frame}
  \frametitle{Example}
\begin{itemize}
  \item From the maximum matching in the previous example we construct the flow network shown below where the maximum matching corresponds to the maximum flow.
\end{itemize}
 \begin{figure}
   \centering
       \tikzstyle{round} = [draw,circle,minimum size=20]
    \begin{tikzpicture}[scale=0.7]
      \node (L) at (-5,8) {$L$};
      \node (R) at (-2,8) {$R$};
      \node[draw,circle](s) at (-9,3) {$s$};
      \node[draw,circle](t) at (2,3) {$t$};
      \draw[->](s) edge[ultra thick] (v1);
      \draw[->](s) edge[ultra thick] (v3);
      \draw[->](s) edge[ultra thick] (v4);
      \draw[->](s) edge (v8);
      \draw[->](s) edge[ultra thick] (v9);


      \draw[->](v2) edge[ultra thick] (t);
      \draw[->](v5) edge[ultra thick] (t);
      \draw[->](v6) edge[ultra thick] (t);
      \draw[->](v7) edge[ultra thick] (t);
      


      \node[style=round] (v3) at (-5,4.5) {v3};
      \node[style=round] (v4) at (-5,3) {v4};
      \node[style=round] (v8) at (-5,1.5) {v8};
      \node[style=round] (v9) at (-5,-0.5) {v9};
      \node[style=round] (v1) at (-5,6) {v1};
      \node[style=round] (v2) at (-2,5.5) {v2};
      \node[style=round] (v5) at (-2,4) {v5};
      \node[style=round] (v6) at (-2,2) {v6};
      \node[style=round] (v7) at (-2,0.5) {v7};
      \node (label1) at (-3,-1.5) {flow network};
      \draw[->]  (v1) edge[ultra thick] (v2);
      \draw[->]  (v3) edge (v2);
      \draw[->]  (v4) edge[ultra thick] (v5);
      \draw[->]  (v3) edge (v5);
      \draw[->]  (v3) edge[ultra thick] (v6);
      \draw[->]  (v4) edge (v6);
      \draw[->]  (v4) edge (v7);
      \draw[->]  (v8) edge (v6);
      \draw[->]  (v9) edge[ultra thick] (v7);
      \draw[->]  (v9) edge (v6);
      \end{tikzpicture}
    
 \end{figure} 

\end{frame}
\end{document}


%%% Local Variables:
%%% mode: latex
%%% TeX-master: t
%%% End:
 